\chapter*{Foreword}
Images and videos have acquired a big prevalence in our daily lives. From the creation of the worlds first camera, the \textit{Camera obscura}, to the creation of one of NASA's most successful science missions, the \textit{Hubble Space Telescope} - it is clear that humans have always had the urge to capture images and use them for scientific and creative purposes. Being able to create machines that can see and differentiate objects in a similar fashion to humans, has revolutionized our lives. Something as simple as recycling bottles in a supermarket is made possible by programming and utilizing techniques from image processing. Doing this enables the bottle refund machine to recognize characteristics of the bottle, rejecting or accepting it depending on the information extracted. This happens all the time wherever we go, but we rarely notice it.

Magic Canvas is a 3$^{rd}$ semester project at Medialogy, Aalborg University, that ran from September to December 2012.

Group MTA 12338 would like to thank Martin J{\o}rgensen and Tone Lunden from Hj{\o}rring Library. We also want to thank our two supervisors: Thomas B. Moeslund and Andreas M{\o}gelmose.