\chapter{Johannes Work}
\chapter{Design}
\section{Introduction to design}
As it was previously decided to create some graphical interface for the interactive installation, using Unity\fixme{Did we tell about Unity earlier?}, some considerations was to be made in regards to the layout. This chapter will focus on the decisions made, to create a smooth interface and a good user experience. In order to create a smooth interface that would excite people in using the setup at the library, a background, some character and some interaction was to be made.

\section{The scene}
The first step into developing a good looking set was to develop a suitable background that ultimately would create a Christmas-like mood. The original idea was to have a few different sets to choose from, depending on the time of the day or week. However initially one background was created to run the prototype testing on. The first background created consist of a static winter landscape, together with some dynamic movable trees which will be described later.\fixme{remember to describe the trees!} \\
In addition; due to light conditions at the library the colors chosen for the background are not all bright, but slightly soften. This action was taken in order for the projector to output a usable illustration. The scene is illustrated on  figure \eqref{fig:ip_Background}.

\begin{figure}[htbp]
\centering
\includegraphics[width=1.00\textwidth]{Pictures/Design/Background.png}
\caption{Image illustrating the initial background for the Christmas game}
\label{fig:ip_Background}
\end{figure}

\section{Avatars}
In addition to the background created for the illustration at the library, some avatars were to be made. Considerations towards creating suitable avatars were made and it was decided to create a pixie boy- and girl, a snowman, an angle and a squirrel.
\subsection{Self-drawn}
It was decided to draw the avatars by hand instead of using some premade drawings from the internet. The avatars should then be imported as assets in Unity. \\
To start with outlines of the characters was drawn on sheets of paper. When the outlines was satisfying they were imported to the computer and drawn again inside Photoshop using drawing tablets. The drawing tablets has pressure sensor which makes the lines thickness chance accordingly to how hard you push. This brings life into the characters and makes them more interesting to look at. At last the characters was coloured, to make a bigger contrast to the background, and to make it more appealing to the audience.\\
Considering the primary group of people trespassing the setup at the library, it was decided to give the avatars a cartoonish look. This was done by using a variety of techniques. One of them was to create oversized body parts e.g. oversized heads. Two of the characters; a pixie girl and an angel is shown on figure XX and XX

