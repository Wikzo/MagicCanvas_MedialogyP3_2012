\chapter{Target Group Analysis}

In this chapter we will investigate the target group of the project. A preliminary assessment will be made, and we will gather information that will help towards the ultimate goal of identifying any special requirements the target group demands. This will consist of three steps; Identifying our target group, describing the target group and compiling and concluding our results.

\section{Identifying out target group}
The preliminary identification of the target group is easily accomplished in this project, as the final installation will be available for all visitors at Hj{\o}rring Library. Upon visiting the library it becomes apparent that \textit{visitors} covers a very large demographic. There are people ranging in age from toddlers, to senior citizens and everything in between. To avoid limiting the installation to a specific part of the visitors at the library, the target group was kept as \textit{all visitors}.
\fxnote{we might want to expand a bit on this}

\section{Description of Target Group}
During a small series (sample size was 22 people) of qualitative interviews the group gathered information about the visitors at the library. We were able to gather some information, and use this to create 4 groups that we will use to categorize our target audience:

\begin{itemize}

\item \textbf{Group 1} \textit{Small children}, the youngest group, were there because their parents enjoyed utilizing the facilities that the library offers.

\item \textbf{Group 2} \textit{Teens and youngsters} enjoyed spending time at the library while waiting for busses, this gave them the opportunity to enjoy each others company and make use of the computer, video games etc. that the library offers.

\item \textbf{group 3} \textit{Students} used the library for work, finding literature for reports and writing there.

\item \textbf{Group 4} Most \textit{adult visitors} used the library to rent material, and not much else.

\end{itemize}

As well as the interviews, we made several observations during the visit. The most active part (using the facilities provided by the library) of visitors at the library was by far the younger part. While the very youngest children were there with their parents, and would not benefit from an interactive installation. It is our belief that the teens and youngsters that use the facilities already offered by the library; computers, video games consoles and the other interactive toys \fxnote{maybe find a better wording for other interactive toys} will be the group that would enjoy an additional interactive installation the most.

Generally the tone at the library is that of any other library, and as such the visitors, mainly students and adult visitors naturally expect a quite and calm working area.

\section{Compilation and conclusion}

It is clear that the visitors at the library are a diverse group, that all use the library in different ways. When considering an interactive installation in the library, the group that would have the greatest interest in such a thing, is the group that also currently use the most of the unique facilities offered by the library; the teens and youngsters. But it, as mentioned before, important for us to accommodate all groups, so as not to exclude anyone from the audience or end up with too narrow a focus.
\\
While we state that a particular group might spend more time on the product, it is important for us to accommodate the other visitors as well. We will not target teens and youngster specifically, but there will be certain affordances that likely will appeal more to this group than the others.
\\
It is also a goal for the finished project fit into a library without disturbing visitors that to not wish to interact with the installation. This means that the final product will not having any sounds that causes more disturbance than someone walking by, coughing or having a normal conversation - as you would expect in a library.
\fxnote{what do we end up with? any sounds at all, correct this part to correspond to the final product. Might also be dangerous writing this, as you could argue that we might have to categorize sounds}
\\
In order for the installation to be a fun experience for both a casual by-passer as well as a more interested user, the final product will function on more than one level, this means that you can pass by and admire the installation and you will also be able to spend more time in front of it.