\chapter{Target group analysis}
In this chapter we will investigate the target group for the project. A preliminary assessment will be made to gather information that will help towards making the program suit the target group best. This will consist of three steps:

\begin{enumerate}
\item Identifying the target group.
\item Describing the target group.
\item Compiling and concluding on the results.
\end{enumerate}

\section{Hj{\o}rring Library: a place for experiences}\label{hjoerring}

%%%% I restructured the whole paragraph, but all the main points are still there. - Marta %%%%
Hj{\o}rring Library aims to distinguish itself from most other libraries. The first distinguishing factor is the location of the library: a big shopping mall. Inside, visitors find themselves in a big open hall with modern architecture. The different sections are divided by bookshelves and a red stripe that runs throughout the library (called "den r{\o}de tr{\aa}d"). While the library first and foremost is a library as any other, one of the prevalent feelings that visitors get is that Hj{\o}rring Library is a place to "hang out" and relax (see figure \ref{fig:p1}), but also a place to have fun or do some work. There are many unique pieces of furniture, creating a great environment for all purposes. Especially kids can have a fun time playing games (physical or digital) and record videos in the children zone (see figures \ref{fig:p2} and \ref{fig:p3}).

At Hj{\o}rring Library the staff has made a big effort to create an environment as interactive as possible, with many places for people to entertain themselves with other things than just traditional books. Even the cafeteria appears to be part of the library's ordinary environment where people can show up, eat their lunch, drink coffee or just have a beer after work.

\begin{figure}[htbp]\centering
	\begin{minipage}[b]{0.3\textwidth}\centering
		\includegraphics[width=1.00\textwidth]{Pictures/HjoerringLibrary/p1.jpg} %Venstre billede
	\end{minipage}\hfill
	\begin{minipage}[b]{0.3\textwidth}\centering
		\includegraphics[width=1.00\textwidth]{Pictures/HjoerringLibrary/p2.jpg} %Venstre billede
	\end{minipage}\hfill	
	\begin{minipage}[b]{0.3\textwidth}\centering
		\includegraphics[width=1.00\textwidth]{Pictures/HjoerringLibrary/p3.jpg} %Højre billede
	\end{minipage}\\ %Captions and labels
	\begin{minipage}[t]{0.3\textwidth}
		\caption{Lounge.} %Venstre caption og label
		\label{fig:p1}
	\end{minipage}\hfill
	\begin{minipage}[t]{0.3\textwidth}
		\caption{Playing games.} %Venstre caption og label
		\label{fig:p2}
	\end{minipage}\hfill	
	\begin{minipage}[t]{0.3\textwidth}
		\caption{Physical Angry Birds game} %Højre caption og label
		\label{fig:p3}
	\end{minipage}
\end{figure}

Hj{\o}rring Library gets more than 1000 visitors every day \fxnote{refer to Martin as a primary source}. A way to maintain this number of visitors is by giving them a new and exiting experience every time they visit. One way to achieve this is by having various themes that run throughout the whole library and changing them every six weeks. These includes topics such as: Arabia, birds, fairy tales or simply the color brown. In the period of this project Hj{\o}rring library has chosen the topic of Christmas, and it will run from the second week of November until the last week of December, 2012. At the time of the first visit to the library, a bird-theme was running. All kinds of birds were exhibited around the library (see figures \ref{fig:b1} and \ref{fig:b2}), and bird songs were playing on several hidden speakers.

\begin{figure}[htbp]\centering
	\begin{minipage}[b]{0.48\textwidth}\centering
		\includegraphics[width=1.00\textwidth]{Pictures/HjoerringLibrary/b1.jpg} %Venstre billede
	\end{minipage}\hfill
	\begin{minipage}[b]{0.48\textwidth}\centering
		\includegraphics[width=1.00\textwidth]{Pictures/HjoerringLibrary/b2.jpg} %Højre billede
	\end{minipage}\\ %Captions and labels
	\begin{minipage}[t]{0.48\textwidth}
		\caption{Birds.} %Venstre caption og label
		\label{fig:b1}
	\end{minipage}\hfill
	\begin{minipage}[t]{0.48\textwidth}
		\caption{More birds.} %Højre caption og label
		\label{fig:b2}
	\end{minipage}
\end{figure}

These themes are an example of how Hj{\o}rring Library tries to work with the concept of \textit{serendipity}, which basically describes a happy accident, something you didn't expect but turned out to be a pleasant surprise. Even though you go to the library to find a book, you might also experience and learn something that you originally did not expect.

Hj{\o}rring Library is always looking for new projects to involve its visitors in new and engaging ways. Since some Medialogy students previously came to the library to do projects and they turned out to be successful, the staff is eager to try it out again. This is where our group comes into the picture. To us, this is a great opportunity to collaborate with an institution on a project in a "real-life" setting, where people outside of the university will interact with the product. Instead of only testing under artificial circumstances, the library will allow us to test on visitors at the library throughout December.

\subsection{Visitor data and placement of the canvas}
The library is divided into several parts, each aiming to give a different experience. As mentioned earlier in section \ref{sec:collab}, it was important for the group to get a central location for the placement of the canvas. This would ensure that a lot of people would have the chance to interact with the program. In fact, the bookshelves chosen are just in the middle of the library, so it is almost impossible not to walk past it at some point (see figure \ref{fig:library_floorplans}).

\begin{figure}[htbp]
\centering
\includegraphics[width=0.80\textwidth]{Pictures/HjoerringLibrary/hjoerring_library_floorplans.png}
\caption{Hj{\o}rring Library is divided into multiple sections. We chose to use a big four-sided bookshelf in the core of the library. Image inspired by \citep{hjoerring_study}.}
\label{fig:library_floorplans}
\end{figure}

%%%%%%%%%%%%%%%%%%%%%%% \citep{hjoerring_study} is shown as [hjo]!!!!!!!!!!!!!!!! %%%%%%%%%%%%%%%%%%%%%%%%

In 2010 \citep{hjoerring_study} did a survey to investigate people's habits and movement patterns at Hj{\o}rring Library. They found that an average visitor spends about half an hour at the library. Those in the age group between 0-10 years and 21-30 years spent the most time at the library, 34 and 44 minutes, while those between 41-50 and 61-70 years spent 19 and 26 minutes during an average visit.

Several cylinder and flow maps were made to visualize the gathered data. Two are shown in figures \ref{fig:library_cylindermap} and \ref{fig:library_flowmap}.

\begin{figure}[htbp] \centering
\begin{minipage}[b]{0.45\textwidth} \centering
\includegraphics[width=1.00\textwidth]{Pictures/HjoerringLibrary/library_cylinder_diagram_24Nov.png} % Venstre billede
\end{minipage} \hfill
\begin{minipage}[b]{0.45\textwidth} \centering
\includegraphics[width=1.00\textwidth]{Pictures/HjoerringLibrary/library_flow_Nov21.png} % Højre billede
\end{minipage} \\ % Captions og labels
\begin{minipage}[t]{0.45\textwidth}
\caption{Cylinder map showing multiple visitor's accumulated visiting time at Hj{\o}rring Library Tuesday November 24, 2010. Image created by \citep{hjoerring_study}.} % Venstre caption og label
\label{fig:library_cylindermap}
\end{minipage} \hfill
\begin{minipage}[t]{0.45\textwidth}
\caption{Flow map showing a single visitor's movement between 10.20-11.15, Saturday November 21, 2010. Image created by \citep{hjoerring_study}.} % Højre caption og label
\label{fig:library_flowmap}
\end{minipage}
\end{figure}

\section{Identifying the target group}
As the final installation will be available for all visitors at Hj{\o}rring Library, the preliminary identification of the target group is easily accomplished in this project. Upon visiting the library it becomes apparent that the visitors cover a large demographic. There are people ranging in age from toddlers, to senior citizens and everything in between. To avoid limiting the installation to a specific part of the visitors at the library, the target group was kept as all visitors. And since the program developed during this project is meant as an art installation, and not as a practical tool, it seemed unnecessary to limit the project to narrower target group. That being said, it would be easy to assume that the program would appeal to children, especially when having the Christmas theme in mind. However, there should be no reason for any adult not to try it out as well.

\section{Description of the target group}
During a small series of qualitative interviews with 22 people at the library, some information was gathered about the visitors. Although the sample size is small, we used the information to create four groups that categorize the target audience:

\begin{itemize}

\item \textbf{Group 1} \textit{Small children}, the youngest group were there because their parents enjoyed utilizing the facilities that the library offers. They liked to play around in the kids zone.

\item \textbf{Group 2} \textit{Teens and youngsters} enjoyed spending time at the library while waiting for busses. This gave them the opportunity to enjoy each others company and make use of the computers, games, books and other facilities at the library.

\item \textbf{Group 3} \textit{Students} used the library for work, finding literature for reports and writing there. The library has various study rooms that can be used when a quiet place is needed.

\item \textbf{Group 4} Most \textit{adult visitors} used the library to rent material, and not much else. Some people also came to drink coffee at the cafeteria.

\end{itemize}

As well as the interviews (see appendix \ref{targetgroup_visit}) several observations were made during a visit that lead us to believe that the teens and youngsters who already use the facilities offered by the library (games, interactive toys, films, etc.) will be the group that would enjoy an additional interactive installation the most.

Generally the atmosphere at the library is the same as that of any other library, as such the visitors (mainly students and adult visitors) naturally expect a quiet and calm working area. This means that the program shouldn't make too much noise.

\section{Compilation and conclusion}

It is clear that the visitors at the library are a varied group of people that all use the library in different ways. When considering an interactive installation, the group that would have the greatest interest in such a thing, is the group that uses most of the unique facilities offered by the library: The teens and youngsters. It is however, as mentioned before, important for us to accommodate all groups.

The goal for the finished program is to fit into the library without disturbing visitors who do not wish to interact with the installation. This means that the final product will not have loud sounds that disturb visitors.

In order for the installation to be a fun experience for both a passive- and an active user, the final program will function on more than one level. This means that you can pass by and admire the installation, but you will also be able to spend more time in front of it.