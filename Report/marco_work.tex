\chapter{Marcos WORKS!}

\subsection{8-bit image and grayscale images}
In an 8-bit image, the 8-bit prefix describes the \textit{bit depth} of the image. The bit depth tells us about the amount of information that can be stored in a single pixel.

An image with a single channel of information for each pixel in the X and Y axis is typically a grayscale image, since each pixel is limited to information about a single hue. The hue defines how pure and vivid a color is \citep{visual_story}. The information stored in each pixel is the brightness of the particular pixel. 8 bit evaluates to $2^8$ different states, meaning that a single 8-bit pixel can display 256 different levels of intensity, or in the case of a zero-based computer system, 0-255 (see figure \ref{fig:ip_grayscale}). An 8-bit image is a widely-used format for images, but it is also possible to create images with more depth and with more information for each individual pixel. Examples of that are images in 16 and 32 bits. A 16-bit depth is equal to $2^{16}$ different states, which means that each pixel can hold any one of 65,536 different shades of gray. Simultaneously, a 32-bit depth is equal to $2^{32}$, which gives 4,294,967,296 different shades of gray.
