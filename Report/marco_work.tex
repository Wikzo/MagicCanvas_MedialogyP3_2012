\chapter{Marcos WORKS!}

\subsection{8-bit image and grayscale images}
In an 8-bit image, the 8-bit prefix describes the \textit{bit depth} of the image. The bit depth tells us about the amount of information that can be stored in a single pixel.

An image with a single channel of information for each pixel in the X and Y axis is typically a grayscale image, since each pixel is limited to information about a single hue. The hue defines how pure and vivid a color is \citep{visual_story}. The information stored in each pixel is the brightness of the particular pixel. 8 bit evaluates to $2^8$ different states, meaning that a single 8-bit pixel can display 256 different levels of intensity, or in the case of a zero-based computer system, 0-255 (see figure \ref{fig:ip_grayscale}). An 8-bit image is a widely-used format for images, but it is also possible to create images with more depth and with more information for each individual pixel. Examples of this are images in 16 and 32 bits, capable of holding information about $2^{16}$ and $2^{32}$ different states in each pixel.

\section{Segmentation up in this bitch}

Segmenting an image or video is the process of extracting the information you are interested in from an image. Segmentation is often a compound operation that consists of different sub operations, all with the same goal of getting the information you want from the input image. Before segmentation algorithms are applied an image undergoes preprocessing, this might be converting from a color image to a grayscale image to ease the computations required. Figure \ref{fig:ip_ColoredToGrayscaleToBinary} serves as an illustration of preprocessing and segmentation. First the image is converted to a grayscale image, this step is part of the preprocessing. Then the image is thresholded, which yields a binary image with only the pens and the heart. Further applying segmentation algorithms will help to reach what ever goal the programmer has in mind.