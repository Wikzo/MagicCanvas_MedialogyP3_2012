\chapter{Marcos WORKS!}

\subsection{LED}
For illuminating the scene, several different alternative were discussed and ultimately IR LEDs were chosen \fxnote{Refer to the chapter in which this is discussed} because of their good illumination outside of the visible spectrum. For the project two different LEDs were tested, both emitting at roughly 880nm \fxnote{doublecheck this}. A 3mm with a cone of light of 30 degrees\fxnote{find source on this} and a 2mm with a cone of light of 65 degrees\fxnote{find source on this}. \fxnote{get pictures for both of these}

During initial testing the focussed beam from the 3mm LED looked as though it might be too focussed, making it  difficult to create a bright line of unbroken illumination. For that reason a series of experiments were conducted: A setup that tested the light that the two LEDs offered \fxnote{refer to image} were constructed and was apparant that the broader LEDs did not focus the light enough for the specific application in the project. Therefore the choice was made to go with the 3mm, 30 degree \fxnote{make sure this is the right number of degrees} LEDs.

Initially a broader strip of light was thought to be desirable, therefore a test was conducted where the LEDs had their lens sanded down in order to make the less focuessed. Figure \fxnote{insert correct picture} shows an image where the LEDs in progression from left to right get more and more sanded down. As the figure illustrates the intensity of the light decreases as the LED-lens is sanded down. After conducting these tests,  if became clear that an unaltered LED performed better than the modified ones, providing a more intense illumination.

For ease of construction the LEDs were arranged into arrays of 8 LEDs connected in series, which in turn are connected in parrallel. The LEDs have a forward voltage of 1.5V, which means that the power supply has to be able to supply at least 12V to them. Initially a small AC-DC power supply capable of supplying 600mA was used for testing purposes. However the total draw of the LED array is 0,5A and for the project a total of 4 arrays were planned. The current far exceeded what could be supplied by the small AC-DC suplly. Luckily the group managed to scavange a 300 watt computer powersupply capable of supplying 12 volts at 8 amperes. After getting the power supply to work without being in a computer, terminals were soldered to LED-strips and the power supply allowing the LED-strips to plug into the power supply.


