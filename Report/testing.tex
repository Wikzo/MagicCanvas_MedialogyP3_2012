\chapter{Testing the final program}
\section{Introduction to testing}
Testing is an important part of the project. It is a great opportunity to get an impression of how the users experience the product. It is specially important because the users might be able to find problems that it is hard to find as creator of a product. \\
The test can give different results, but no results is more correct than others. Maybe the users like the project and it is a success. Maybe there are small errors that the users would like to be changed. It is also possible that the users do not like the project at all, and call it a failed attempt. No matter what the result is then the critique is always useful. Since the critique comes directly from the users of the product then they should be taken seriously and all proposals should be taken into consideration. 

The testing of the installation will take place at Hj{\o}rring Library where the product is being exhibited. The goal for this test is to see how people engage to the installation. The focus for the project is to entertain people so it is interesting to get an overview of how this goal is achieved. It is also interesting to see if people understand the concept and understand what the installation is capable of doing. Since it is designed for the users to either just walk by or to engage in the system and play with it, then it is interesting to ask about the experiences from both groups of users. \\ 

\section{Tendencies test}
Running the test will be a bipartite task, as the first test will focus on the visitors, while the second test will focus on the installation.\\
During the first test; one team will focus towards monitoring peoples habits, gender and ages, while another group will interview users. The second test will be a user test where the tracking of people will be in focus.

In regards to observing, two persons from the group will be present at the setup to observe people walking by. One will focus on people using the setup actively and one will focus on people using the setup on a passive level.\\
The members present will take notes in regards to formalities such as gender and age, in order to determine if there are tendencies of a certain group of people who avoid interacting in the setup e.g.\\
In addition the group present at the setup will note down how people react to the installation and how they engage. It's of importance to find pattern in the way people interact and look for tendencies; in order to improve the product. Considering the scenario that three quarters of the people who engage actively in the installing gather around the center and remain static, the program should provide some feedback that will make people move.

\section{Interviews}
The second group of the testers will be stationed of either side of the installation, to ask people who walked by the installation to participate in answering some simple questions. People will be kindly asked if they are interested in giving some feedback to improve the user experience and given a reward in form of "Pebbern{\o}dder" or "brunkager", as a friendly gesture.\\
The idea of placing two members on either side of the installation is in order to gather information after they have had the chance to engage in the setup. The interview will look the following:

\textbf{Interview}
\begin{itemize}
\item Active users
\begin{itemize}
\item How old are you?
\item What is your gender?
\item Why did you visit the library today?
\item Does the installation fit in the theme of the library?
\item Why did you use the installation at the library?
\item What can be done to improve the product and the overall experience?
\end{itemize}
\item Passive users
\begin{itemize}
\item How old are you?
\item What is your gender? 
\item Why did you visit the library today?
\item Does the installation fit in the theme of the library?
\item Why did you not use the installation at the library?
\item What can be done to make you engage in the installation?
\end{itemize}
\end{itemize}

\subsection{Usability testing}
The second task of the testing is to run a user test which main focus will be to evaluate the installation at the library. One of the primary aspect of the installation is the possibility to track several people to apply two or more avatars at once. Therefore it is of great importance that the program works as intended.\\
The test will primarily be concentrated on tracking people and applying the BLOB analysis. The following tests will be run:  
\begin{itemize}
\item Fidelity test
\item Number of simultaneous tracking
\end{itemize}

\section{Analysing the test results}
After going to Hj{\o}rring library to run the different tests, the analyzing part is to be made. The testing ran for an entire day, to see if there were any changes in tendencies during the day. The testing will be evaluated and analyzed in different subsections to focus on all aspects of importance. \\
All test persons were offered some Christmas treats in return for the help. In addition the majority of users were positive and friendly minded. Unfortunately December is a busy month so many people were occupied and didn't have time to participate in the testing. Furthermore there was a low amount of visitors at the library the day the testing occurred. Accordingly to Tone, the librarian at Hj{\o}rring library, December is not the busiest month a the library due to Christmas associated activities.

\subsection{Common tendencies}
Testing the common tendencies was done by observing how people who engaged in the installation. The representatives from the group were stationed away from the installation to avoid interference and to sustain consistency.\\ 
After about an hour it was clear that there was some common tendencies, some more desirable than others. The one tendency that draw the most attention to us, was the fact that people did not notice the installation when they walk by it. This proved a substantial flaw in the creation of the installation and it was considered critical as the essence of the installation is that people notice its appearance.\\
A lot of people just walked by the canvas without even looking at it, while others looked at the canvas, but didn't notice the interactive possibilities. It appears that Hj{\o}rring library is a place where several people are sneaking around quietly minding their own business and not really observing much.\\
Another tendency was that children were more likely to play with the installation compared to adults. It looked like it is more natural for the children to engage and play with the product, and they seem more fascinated by it. An hypothesis could be that it is more natural for children to play. Furthermore; the design is somewhat "cartoonish", which would be considered more apearling to children rather than adults.\\
Commonly it seemed that the children visiting Hj{\o}rring library were more likely to engage in the Christmas exhibitions and decorations around the library.

\subsection{Interviews}
The interviews were made after the visitors have had the chance to engage in the installation, like described in the subsection above. It was decided to interview both passive and active users of the installation in order to compare the submissions on the questions that were identical to both active and passive users. Secondly it was of importance to generate some specific questions that applied for either group, in order to receive some valuable feedback. A total amount of 10 persons were interviewed. The persons ages ranged from 14 to 67 to get opinions from different ages.\\
The gender were equally divided by five male and five female test persons. Out of the 10 testers four users engage actively in the installation, while six engaged on a passive level\\
The main reasons why the persons were visiting the library varied widely but can mostly be divided into two groups. The first group is of persons who came to the library to have fun. Examples could be a 14 year old boy who came to the library to play PlayStation, a 67 years old woman who came all the way from Aalborg to visit the library after reading about it's popularity or a 33 year old woman who came to spend the afternoon with her kids. The other group of people who ventured to the library, was to borrow books. An examples of this is a 15 years old girl who came to borrow the Twilight book and a man aged 39 who brought his kid along to pick up books, which they do often.\\
The fact that people visit the library on so many different occasions gives a great approach for the testing, as it provides several different perspectives.\\
Firstly people who engaged passively in the installation will be evaluated and later on the answers submitted by the active users will be evaluated.\\

\subsubsection{Passive testers}
The passive testers were asked why they did not use the installation. Five out of the six passive users did not use the installation because they did not notice it, which creates a big concern as it is a critical flaw in the creation of the installation. The fact that half of the testers did not even notice the installation is a huge problem, which has to be reassessed if another test was to be done. The only passive tester who actually noticed the canvas didn't use it because he thought of it as childish (He was 19 years old).

Next question was regarding how the installation was fit into the library and the Christmas theme. The majority, 5 out of 6 participants thought that the installation fit the Christmas theme. Furthermore a woman expressed that the installation fit Hj{\o}rring library with excellence, as its interactivity compliment the general interactive element of Hj{\o}rring library well. 

Last question for the passive users were about ideas for improvements, or additions that would make them use the product.
Most of the testers mentioned that there had to be drawn more attention to the installation, so that they would notice it in the first place. Ideas for this could be a sign on the red interior if the library with the text \textbf{Opstilling udf{\o}r af Aalborg Universitet}. Or perhaps a sign on the floor saying that the installation is interactive and that people can engage. Another idea was to put sounds into the program to catch attention. Multiple testers also mentioned that brighter colors would help drawing attention to the canvas as it would create a better contrast and make it more eye-catching. 
One tester (14 year old boy) told that it had to get more features while another said that it had to be less childish (19 year old boy).

\subsubsection{Active testers}
To begin with the active testers was asked what made them use the installation. This gave very different answers. One tester said that that primary reason for using the installation was due to his daughter who's attention was drawn immediately.\\
Other testers said that it was because of the dynamics and because the characters looked cute. However the most general answer is that it is because it different a lot from the rest of the library.

Next question was regarding how the installation fit in the library and with the Christmas theme. All active users said that it fit the theme and the library, and one tester even said that he hoped they would do something similar upcoming years.

Last question focused upon future implementations in regards to improvements that would make the product even better. Two testers said that more movement would improve the quality of the product. Also more interaction would be good, a proposal was to make the user able to touch the screen.
And as mentioned earlier - clearer colors would improve the product significantly because it would make it easier to see.\\
Last but not least a tester mentioned that another location could improve it, as she thought it was embarrassing to walk back and forth again and again, so maybe a more discrete location could be used.

Overall the interviews provided some good feedback on how to improve the overall experience. These matters will be considered in the following chapter, Future Implementations.

\subsection{Usability testing}
%%% INSERT PICTURE OF USABILITY TESTING %%%
%%%  CAN USE SCREEN SHOT FROM AV MOVIE  %%%

The usability testing is made by group itself to see how well the installation works and to see if the final product is satisfying.\\
The installation works well and it is easy to tell that you are controlling the movements of the characters when walking back and forth. The fact that it is random which character you are granted is a good detail. It is a fun addition and it creates some excitement that you don't know which character will appear on the canvas. The installation has around 1 second delay, which is okay, but a faster reaction time would definitely improve the product. The problem problem regarding this matter is that often people don't notice the character at all before they have walked past the canvas, which basically means they don't notice the interactivity or the installation itself. \\
The program should allow more people to use it at the same time which also works most of the time. However every now and then it has some problems detecting both persons if they walk too close as it will look like one person.
The idea of Father Christmas showing up in a random interval is fun, but none of the testers who were interviewed saw the actual event, so perhaps it should occur more often.\\
The whole setup is easy to open and it requires no configurations during that start-up whatsoever.\textbf{The only problem is that small movement caused to the LEDs or the camera will give the program an error so that people is not detected. This could maybe be fixed by isolating the camera and LEDs so no one can touch them.}\fixme{I think that Max already made a function that will check update every hour to avoid this PLEASE CONFIRM}\\
The overall evaluation is that the installation is working really well and the problems that may occur is not too problematic. 

