\chapter{Testing the final program}
\section{Introduction to Testing}
Testing is an important part of the project. It is a great opportunity to get an impression of how the users experience the product. It is specially important because the users might be able to find problems that it is hard to find as creator of a product. \\
The test can give different results, but no results is more correct than others. Maybe the users like the project and it is a success. Maybe there are small errors that the users would like to be changed. It is also possible that the users do not like the project at all, and call it a failed attempt. No matter what the result is then the critique is always useful. Since the critique comes directly from the users of the product then they should be taken seriously and all proposals should be taken into consideration. 

The testing of the installation will take place at Hj{\o}rring Library where the product is being exhibited. The running of the test will be described in the next section.

\section{Running the test}
Running the test will be a bipartite task, as the first test will focus on the visitors, while the second test will focus on the installation.\\
During the first test; one team will focus towards monitoring peoples habits, gender and ages, while another group will interview users and present a questionnaire. The second test will be a user test where the tracking of people will be in focus.

\subsection{Observation}
In regards to observing, two persons from the group will be present at the setup to observe people walking by. One will focus on people using the setup actively and one will focus on people using the setup on a passive level.\\
The members present will take notes in regards to formalities such as gender and age, in order to to determine if there are tendencies of a certain group of people who avoid interacting in the setup e.g.\\
In addition the group present at the setup will note down how people react to the installation and how they engage. It's of importance to find pattern in the way people interact and look for tendencies; in order to improve the product. Considering the scenario that three quarters of the people who engage actively in the installing gather around the center and remain static, the program should provide some feedback that will make people move.

\subsection{Handing out questionnaires}
The second group of the testers will be stationed of either side of the installation, to ask people who walked by the installation to participate in answering some simple  questions. People will be kindly asked if they are interested in giving some feedback to improve the user experience and given a reward in form of "Pebbern{\o}dder or brunkager", as a friendly gesture.\\
The idea of placing two members on either side of the installation is in order to gather information after they have had the chance to engage in the setup. The questionnaire will look the following:

\textbf{Questionnaire}
\begin{itemize}
\item Active users
\begin{itemize}
\item Why did you visit the library today?
\item Does the installation fit in the theme of the library?
\item Why did you use the installation at the library?
\item What can be done to improve the product and the overall experience?
\end{itemize}
\item Passive users
\begin{itemize}
\item Why did you visit the library today?
\item Does the installation fit in the theme of the library?
\item Why did you not use the installation at the library?
\item What can be done to make you engage in the installation?
\end{itemize}
\end{itemize}

\subsection{Running a user test}
The second task of the testing is to run a user test which main focus will be to evaluate the installation at the library. One of the primary aspect of the installation is the possibility to track several people to apply two or more avatars at once. Therefore it is of great importance that the program works as intended.\\
The test will primarily be concentrated on tracking people and applying the BLOB analysis. The following tests will be run:  
\begin{itemize}
\item Fidelity test
\item Number of simultaneous tracking
\end{itemize}


