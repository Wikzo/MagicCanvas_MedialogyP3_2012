\chapter{Testing the final program}
\section{Introduction to Testing}
Testing is an important part of the project. It is a great opportunity to get an impression of how the users experience the product. It is specially important because the users might be able to find problems that it is hard to find as creator of a product. \\
The test can give different results, but no results is more correct than others. Maybe the users like the project and it is a success. Maybe there are small errors that the users would like to be changed. It is also possible that the users do not like the project at all, and call it a failed attempt. No matter what the result is then the critique is always useful. Since the critique comes directly from the users of the product then they should be taken seriously and all proposals should be taken into consideration. 

The testing of the installation will take place at Hj{\o}rring Library where the product is being exhibited. The goal for this test is to see how people engage to the installation. The focus for the project is to entertain people so it is interesting to get an overview of how this goal is achieved. It is also interesting to see if people understand the concept and understand what the installation is capable of doing. Since it is designed for the users to either just walk by or to engage in the system and play with it, then it is interesting to ask about the experiences from both groups of users. \\ 

\section{Tendencies test}
Running the test will be a bipartite task, as the first test will focus on the visitors, while the second test will focus on the installation.\\
During the first test; one team will focus towards monitoring peoples habits, gender and ages, while another group will interview users. The second test will be a user test where the tracking of people will be in focus.

In regards to observing, two persons from the group will be present at the setup to observe people walking by. One will focus on people using the setup actively and one will focus on people using the setup on a passive level.\\
The members present will take notes in regards to formalities such as gender and age, in order to determine if there are tendencies of a certain group of people who avoid interacting in the setup e.g.\\
In addition the group present at the setup will note down how people react to the installation and how they engage. It's of importance to find pattern in the way people interact and look for tendencies; in order to improve the product. Considering the scenario that three quarters of the people who engage actively in the installing gather around the center and remain static, the program should provide some feedback that will make people move.

\section{Interviews}
The second group of the testers will be stationed of either side of the installation, to ask people who walked by the installation to participate in answering some simple questions. People will be kindly asked if they are interested in giving some feedback to improve the user experience and given a reward in form of "Pebbern{\o}dder" or "brunkager", as a friendly gesture.\\
The idea of placing two members on either side of the installation is in order to gather information after they have had the chance to engage in the setup. The interview will look the following:

\textbf{Interview}
\begin{itemize}
\item Active users
\begin{itemize}
\item How old are you?
\item What is your gender?
\item Why did you visit the library today?
\item Does the installation fit in the theme of the library?
\item Why did you use the installation at the library?
\item What can be done to improve the product and the overall experience?
\end{itemize}
\item Passive users
\begin{itemize}
\item How old are you?
\item What is your gender? 
\item Why did you visit the library today?
\item Does the installation fit in the theme of the library?
\item Why did you not use the installation at the library?
\item What can be done to make you engage in the installation?
\end{itemize}
\end{itemize}

\subsection{Usability testing}
The second task of the testing is to run a user test which main focus will be to evaluate the installation at the library. One of the primary aspect of the installation is the possibility to track several people to apply two or more avatars at once. Therefore it is of great importance that the program works as intended.\\
The test will primarily be concentrated on tracking people and applying the BLOB analysis. The following tests will be run:  
\begin{itemize}
\item Fidelity test
\item Number of simultaneous tracking
\end{itemize}

\section{Analysing the test results}
After going to Hj{\o}rring library to run the different tests the analysing part is being made. The testing was running for a whole day, to see if there were any changes in tendencies during the day. The testing will be evaluated and analysed in subsections to put focus on all important aspects. \\
All test persons was granted with Christmas treats and the majority of users were positive and gladly helped. Unfortunately December is a busy month so many people were busy and didn't have time to participate in the testing. Furthermore there were a low amount of visitors at the library the day the testing occurred.

\subsection{Common tendencies}
Testing the common tendencies was done by observing how people engage in the project and play with the installation from distance - not to effect the results.\\ 
Doing this highlighted some clear tendencies. The biggest tendency is that people do not notice the installation when they walk by it. This is a critical problem because it is impossible for people to use the installation if they doesn't even notice it.\\
A lot of people just walk by the canvas without even looking at it, while others look at the canvas, but doesn't notice its possibilities. It seems as if the library is a place where many people walk around quietly thinking and not really observing much.\\
Another tendency is that children are more likely to play with the installation compared to adults. It looks like it is more natural for the children to engage and play with the product, and they seem more fascinated by it. This could be because it is more natural for children to play. Another reason could be that the design is very "cartoonish", which could be more appealing to children.\\
It also seems as if children are more likely to observe and play with the different Christmas decorations around the library.

\subsection{Interviews}
The interviews were made after the test persons had a chance to use the installation. It is interesting to interview both active and passive users, because they might have different ideas for improvements. The test persons didn't know about the interview till after walking by the canvas using it either passive or active. This was to not affect their opinions and experience of the installation.

A total amount of 10 persons were interviewed. The persons ages ranged from 14 to 67 to get opinions from different ages.\\
The gender were equally divided with five male and five female test persons. Out of the 10 testers four used the installation, while six used it passively.\\
The reasons why the persons were visiting the library varied widely but can mostly be divided into two groups. The first group is of persons who come to the library to have fun. Examples could be a 14 year old boy who came to the library to play PlayStation, a 67 years old woman who came all the way from Aalborg to visit the library after reading about it's popularity or a 33 year old woman who came to have fun with her kids. The other group is of persons who come to the library to borrow books. Examples of this could be a 15 years old girl who came to borrow the Twilight book, or a 39 year old man and his kid who came to pick up books which they do often.\\
The fact that the test persons visit the library of different purposes is great for the test, since it provides opinions from different perspectives.\\
First the persons who used the installation passive will be evaluated, and later in the section the active users answers will be evaluated.\\

\subsubsection{Passive testers}
The passive testers were asked why they didn't use the installation. Five out of the six passive users didn't use the installation because they didn't notice it, which is a big concern. The fact that half of the testers didn't even notice the installation is a huge problem which has to be reassessed if another test was to be done. The only passive tester which actually noticed the canvas didn't use it because he thought it looked too childish (He was 19 years old).

Next question was regarding how the installation was fitting into the library and the Christmas theme. The majority - 5 of 6 - thought that the installation was fitting well into the Christmas theme. In addition one person told that she thought it was fitting perfectly here in Hj{\o}rring because if it's interactive environment. Another person mentioned that it was positioned very well because it was on the way to the kids section.
The last person didn't know what to answer.

Last question for the passive users were about ideas for improvements, or additions that would make them use the product.
Most of the testers mentioned that there had to be drawn more attention to the illustration, so that they would notice it in the first place. Ideas for this could be a sign on the red furniture or maybe on the ground. Another idea was to put sounds into the program to catch attention. Multiple testers also mentioned that brighter colors would help drawing attention to the canvas as it would stand out better.
One tester (14 year old boy) told that it had to get more features while another said that it had to be less childish (19 year old boy).

\subsubsection{Active testers}
To begin with the active testers was asked what made them use the installation. This gave very different answers. One tester said that it was because it drawn his daughters attention. Other testers said that it was because it moved, and because the characters looked cute. However the most general answer is that it is because it is very different from what you usually see in a library.

Next question was regarding how the installation fits in the library and with the Christmas theme. All active users agreed that it fits the theme and the library and one tester even said that he hoped they would do something similar all year.

Last question asked for improvements that would make the product even better. Two testers said that more movement would improve the quality of the product. Also more interaction would be good - a proposal was to make you able to touch the screen.
As others mentioned earlier as well more clear colors would also improve the product because it would make it easier to see. Lastly a tester mentioned that another location could improve it, because she thought it was embarrassing to walk back and forth again and again, so maybe a more discrete location could be used.

The interviews gave some great answers which should be thought through when discussing future implementations.   
  
\subsection{Usability testing}
%%% INSERT PICTURE OF USABILITY TESTING %%%
%%%  CAN USE SCREEN SHOT FROM AV MOVIE  %%%

The usability testing is made by our self to see how well the installation works, and to see if the final product is satisfying. \\
The installation works well and it is easy to tell that you are controlling the characters movement when walking back and forth. The fact that it is random which character you get is really good. It is a fun addition and it makes it more fun that you don't know which character you will appear as before you enter. The installation has around 1 second delay which is okay, but faster reaction time would definitely improve the product. The problem with this is that often people doesn't notice the character at all before they have walked past the canvas, and then they miss the concept. \\
The program should allow more people to use it at the same time which also works most of the time. However every now and then it has some problems detecting both persons if they walk too close as it will look like one person only.
The idea of Santa showing up in a random interval is fun, but none of the testers were lucky enough to see it, so maybe it should occur more often.\\
The whole setup is easy to open and it requires no work from persons. The only problem is that small movement of the LEDs or the camera will give the program an error so that people is not detected. This could maybe be fixed by isolating the camera and LEDs so no one can touch them.\\
The overall evaluation is that the installation is working really well, and the problems that may occur is not too problematic. 

