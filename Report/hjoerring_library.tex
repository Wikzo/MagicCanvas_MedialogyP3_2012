\section{Hjørring Library}
Hjørring Library aims to be something different than most other Danish libraries. This is seen by the fact that it is located inside a big shopping mall. Among other things the library has its own cafeteria and bar. The architecture also appears very modern; everything is in one big room, and there are many places to people to entertain themselves with other things than just traditional books. Hjørring Library wants to a place where people can "hang out". There are many unique pieces of furniture, and especially kids can have a fun time playing games (physical and digital) and record videos.

Hjørring Library approximately gets 1000 visitors every day, and the goal is to give them a new and exiting experience every time. One way to achieve this is by having various themes that run throughout the whole library, changing every six weeks. These includes topics such as: Arabia, birds, fairy tales and brown. The library is built on the idea of serendipity, which basically describes a happy accident; something you didn't expect but turned out to be a pleasant surprise.

Hjørring Library is always looking for new projects that involves its visitors in new and engaging ways. Previously they have had Medialogy students come and do projects, and it turned out to be a success, so they are eager to try it out again. This is were our group comes in. We see it as a great opportunity to try making a project in a "real-life" setting where people outside of the university will engage with our product.

In the period of this project Library Hjørring has chosen the topic of Christmas. This theme runs from the second week of November until the last week of December, 2012.

\section{Interactive expo}

\section{Target group analysis}
Maybe?