\section{Camera theory}

An image captured through a camera is the result of reflected light being detected on a camera sensor, see figure \ref{fig:light_cam}. This process is know as \textit{image acquisition} and is generally not something you think about when you capture an image with a camera, since it all happens automatically. In this section the basics of image acquisition using a digital camera will be explained. But in order to understand how this works it is necessary to have a basic understanding of the physics behind light.

\begin{figure}[htbp] 
\centering 
\includegraphics[width=0.5\textwidth]{Pictures/Theory/light_from_sun.png} 
\caption{Light as captured by a camera} 
\label{fig:light_cam} 
\end{figure}

Light is a form of electromagnetic radiation that can be viewed as both waves and particles. This duality is however not something that will be covered within this chapter, as the wave model is sufficient to build the foundation of the understanding we need. A light wave is a small packet of energy travelling through space. These energy packets are known as photons. Photons can be described by three properties:

\begin{itemize}
\item \textbf{Wavelenght} - Measured in meters from wave top to wave top and denoted as $\lambda$.
\item \textbf{Frequency} - Measured in oscillations per second, Hz, denoted $f$.
\item \textbf{Energy} - Measured in electronvolts, eV, denoted $E$.
\end{itemize}

The formulas for these properties are as follows:
\begin{itemize}
\item To derive the wavelength or the frequency, formula \ref{eq:wavelenght} is applied:
\begin{align}
\centering 
\lambda = \frac{C}{f}
\label{eq:wavelenght} 
\end{align}
where {$C$} is the speed of light.
\item The second equation, equation \ref{eq:e_v}, describes the energy:
\begin{align}
\centering
E = \frac{hC}{\lambda}
\label{eq:e_v} 
\end{align}
Where {$h$} is Planck's constant, which describes the proportional relationship between energy of a photon and an the corresponding electromagnetic waves frequency.
\end{itemize}

The wavelength of the photon determines what color will be perceived. However, as known from physics, the speed of light is constant; thus changing the frequency will alter the wavelength and therefore the color perceived. The visible spectrum of light is only a fraction of the full spectrum of electromagnetic radiation, see figure \ref{fig:em_rad}. Light interacts additively, with a mix of equal parts of each wavelength resulting in white light. The light from the sun can be broken down into its component wavelengths by refracting the light in a prism. This yields the full spectrum of visible light, as seen in a rainbow.
While different wavelengths correspond to different colors, these colors only remain defined within our perception. The reason we see ~600 nm light as yellow is because our eyes contain three types of optically sensitive cells, known as \textit{cones}, sensitive to three colors: red, green and blue. 600 nm lies between the green and red cells, thus both cells are firing and the perceived color becomes yellow.\citep{perception_book}.

\begin{figure}[htbp] 
\centering 
\includegraphics[width=1\textwidth]{Pictures/Theory/em_rad.png} 
\caption{The full spectrum of electromagnetic radiation.} 
\label{fig:em_rad} 
\end{figure}

\subsection{Digital image acquisition}
When capturing an image using a digital camera, light passes through a lens onto the sensor. This acts in much the same way as the human eye: in place of cones sensitive to specific wavelengths, a camera sensor has a physical matrix of pixel sensors (one for each pixel in the output image). A camera that can only capture black and white images has only one type of sensor in each physical pixel whereas a camera that captures color images has three types of sensors in each pixel \citep{ip_book}.

\subsection{Image acquisition in the project}
In the project, two cameras were considered: a normal RGB webcam and a similar camera with an infrared (IR) filter installed. The filter consists of a small strip of normal analogue film, used for analogue pictures, that has been mounted on the lens. The film strip allows only infrared light to pass, thus making the only light that reaches the sensor infrared.

\begin{figure}[htbp] 
\centering 
\includegraphics[width=0.5\textwidth]{Pictures/Theory/IR_filter.png} 
\caption{The IR filter only allows infrared light to pass.} 
\label{fig:ir_filter} 
\end{figure}

This is interesting, as it enables a larger degree of control over the illumination in the scene that is captured. Normally in a scene, controlling the lighting \citep{ip_book} can prove a challenge, and in order to operate without error an image processing algorithm would have to take into account the variance in lighting that occurs naturally during a day cycle. Working with IR eliminates this issue.

\subsection{Infrared light}

Infrared light lies right next to visible light in the electromagnetic spectrum. Looking at the full spectrum in figure \ref{fig:em_rad} 
\textbf{I think we should have a section about IR light, maybe here? - Gustav}

\subsection{Illuminating the scene}

It is important that the lightning in the scene is constant. Imagine that the scene is captured with a normal webcam: this creates an RGB image (see figure \ref{fig:scene_light}). In this case, we imagine a program that is intended to find the shelves. In order to speed things up, the image is converted to grayscale. In the right picture the program can easily find the shelves, but in the picture on the left, the program cannot find the shelves, see figure \ref{fig:scene_thresholded}.

\begin{figure}[htbp] 
\centering 
\includegraphics[width=0.5\textwidth]{Pictures/HjoerringLibrary/scene_lighting.png} 
\caption{The difference in illumination on different times of day is clearly visible when compared next to each other} 
\label{fig:scene_light} 
\end{figure}

\begin{figure}[htbp] 
\centering 
\includegraphics[width=0.5\textwidth]{Pictures/HjoerringLibrary/scene_lighting_thresholded.png} 
\caption{The same image converted to 8bit grayscale and thresholded.} 
\label{fig:scene_thresholded} 
\end{figure}

To avoid running into issues with variances in illumination, the decision was made to illuminate the scene artificially. Different options were evaluated for the illumination, all under the consideration that we had the ability to filter the camera input, allowing only infrared light to pass to the camera. The following options were considered:

\begin{itemize}
\item Using the library's lighting.
\item Making adjustments to the lighting at the library.
\item Bringing normal lights to light the scene.
\item Building our own light system to light the scene.
\end{itemize}

\subsubsection{Using the library's lighting}
The library's lightning consist of two different types of lamps: fluorescent lights, and very bright halogen spots. Using the infrared camera, we looked at the normal lighting and the halogen spots radiated too much IR light for us to be able to have a decent contrast between background and the subject we want to track.

\subsubsection{Making adjustments to the lighting at the library}
It was not possible to adjust the fluorescent lights, but the halogen spots could be pointed in a different direction. We further considered using some material to direct the cone of light from the spots, as to create the largest contrast between background and subject. Two options presented themselves: point the lights at the background, or point the lights to the subjects. These two cases offer two different problems. Limiting the cone of the light to only hit the subjects would require the light to be incident from a angle where they could not be placed, on the other hand limiting the spots to illuminating the canvas would interfere with the projection of images, as the spots are very bright.

\subsubsection{Bringing normal light to the scene} 
By bringing normal lights that emit a high amount of IR lighting, the subject could be illuminated in a sufficient manner, but this would require a mount to be created or bought.

\subsubsection{Building our own light system to light the scene}
Building a lighting system from scratch to light the scene would allow a tailored solution for the installation, removing the problems offered by the other solutions. By having a strip of bright infrared lights, and using this to light up the background, we should be able to create a high contrast between the background and the subject that we want to track. Furthermore, this provides us with the added benefit of already having defined a region of interest, thus easing the workload of the image processing.

\subsection{Concluding on the setup}
Using the default setup at the library did not provide a decent scene to be captured and processed by the program. The lights at the library were too diffuse, making adjustments to the lighting still presented a problem. While the light could be controlled to only illuminate the background, the visible light emitted would shine too bright, making the image from the project hard to make out. Bringing normal lights result in the same problem. This meant that the final decision was to create a custom infrared lighting for the library.
