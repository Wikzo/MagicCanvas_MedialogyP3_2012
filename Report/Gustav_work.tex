\chapter{Problem definition - GUSTAVs edition}
\section{Foreword}
One aspect of Medialogy is building systems that react accordingly to humans. There are several technologies and theories that support this. One of these systems is called Visual Computing where Image Processing often is used, e.g. to recognize specific objects and motions. By implementing custom made software, one will be able to monitor a specific target group or process. In addition to this semester project, two mandatory courses on the third semester of Medialogy are of interest to our  project.\\
The first course is called Image Processing and it is primarily based on understanding the theory of a picture and tools to manipulate the pixel values within. Knowledge such as the principle of bits and bytes, the RGB color system, histograms, and using threshold to segment a picture, are some of the tools learned from the course.\\
The second course used for the semester project is Procedural Programming, in which the C++ programming language was taught. The theoretical background for programming is necessary to be able to analyze and manipulate a digital image, based on theory and algorithms learned in Image Processing.

To structure the semester projects in regards to supervisors and co-supervisors, every group were asked to discuss which subject field that was of interest to them. In continuation of this every group had to to submit an application for the three subjects of the biggest interest, with first, second and third priority.\\
The eleven topics presented were:

\begin{itemize}
\item Image Processing for Fun Utilizing an Industrial Robot
\item Image Processing for Ambient Intelligent Robots
\item Interactive Floor
\item Interactive Book
\item Interactive Drawing Game
\item Interactive Arcade Game
\item Emergency system for old people that have fallen
\item Thermal Sock Puppet Show
\item Body Motion Controlled Exercise Game
\item Mobile app for recognizing electric components
\item Hj{\o}rring Library
\end{itemize}

After reading into the different project proposals our group decided which three subject to apply for in prioritized order:

\begin{enumerate} 
\item Hj{\o}rring Library 
\item Interactive Floor 
\item Image Processing for Ambient Intelligent Robots 
\end{enumerate}

\section{Establishing collaboration with Hj{\o}rring Library}

To great delight, the group was chosen to work with Hj{\o}rring Library, which was our first priority. There were no additional predefined project, just that it should be some kind of installation that could run at the library - preferably based on a special theme.

A meeting was arranged with the library, and the group, together with other groups from third and fifth semester of Medialogy, travelled to Hj{\o}rring to meet the library staff. In addition the groups were let loose in the library to check locations for potential projects.\\
After talking to some of the staff, more specific our contact person Martin J{\o}rgensen, some general ideas emerged, so it was decided to go back to the group room at NOVI to do some brainstorming within the group.

The first idea that emerged included scanning of barcodes on books in the library. The concept was to place a big canvas on some bookshelves, and whenever people walked past it, they would see their silhouette projected onto the canvas. On top of their figure would be some kind of hat or other clothes. The  idea was that people would explore the library, and depending on the books gathered and scanned, some specific graphic would appear on the canvas. An example could be scanning a fairy tale written by H.C. Andersen which would then produce a hat on-top of the user. This allowed for two types of usage: people that casually walk past the shelves will see something interesting on the canvas; if they wanted to, they could decide to participate further by going around the library to find the correct books, scan them, and thereby change what would be projected on the canvas next time.

After conferencing with the supervisor, Thomas Moeslund, and the co-supervisor, Andreas M{\o}gelmose, it was decided that the concept needed to be narrowed down to a more specific premise. The supervisors were unsure whether it would be possible to make a program that could  track people and display graphics on top of them (e.g. a hat), as well as scan barcodes from books and extract some useful data (e.g. book genre, fiction or non-fiction, etc.). It was suggested to only focus on the first part of the concept. Instead of trying to do multiple themes, we should focus mainly on a single theme such as fairy tales and hats from this category.

An email was sent to Martin J{\o}rgensen from Hj{\o}rring Library describing the concept and the location we would like to get: a big bookshelf in the center of the library. Martin liked the idea, but had some concerns to identify the period in which the project would fit the library. Hj{\o}rring Library work around specific topics that change every 8-12 weeks. One of the upcoming themes of interest at the library was Christmas, which runs from the 2nd week of November until the last week of December. Another idea then emerged in the group towards creating a project for December. Instead of working with the H.C. Andersen theme, it was decided to work with Christmas, and to add a Christmas hat on-top of visitors of the library. Martin was fond of this idea and so a new meeting was arranged in order to settle on a suitable location at the library for our project.\\
Two members of the group travelled to Hj{\o}rring to participate in the meeting with Martin. The outcome was a more specific plan on the project concerning:

\begin{itemize}
\item The position of the project
\item The equipment that was desired to borrow
\item The light conditions
\end{itemize}

The position of the canvas was decided to be on the walkway from the reception desk to the core of the library (see figure \ref{fig:concept_art}), which Martin believed would be an ideal position for the project as the light condition is controllable. In addition to that, he knew from experience that people tend to use that walkway either to enter or exit the library, as it is the shortest route. Most importantly, it should be a place that people walk past naturally, instead of having to go out searching for a small corner.

\begin{figure}[htbp]
\centering
\includegraphics[width=1.00\textwidth]{Pictures/HjoerringLibrary/LocationJohannesHat.jpg}
\caption{A central location for the canvas was chosen to get Location a Hj{\o}rring Library}
\label{fig:concept_art}
\end{figure}
Figure \ref{fig:concept_art} illustrates the setup at the library. A canvas with the proportions 3 * 2.25 meters will be positioned directly on the bookshelves seen in the background of the picture.

\subsection{The initial goal}
One thing that Martin was interested in, was knowing the ambitions for the project and what to expect. There had already been a conversation within the group about our level of ambitions for the project. Considering the knowledge in programming and image processing, being ready to present the prototype in the beginning of December became a part of the initial goal. Previously, Martin told us that the Christmas theme would run from the second week in November until the last week in December; therefore a 3 week run-time period was a desirable goal.\\
As for the initial goal, Martin expressed his excitement by letting us know that it was very ambitious, but that we should not be intimidated by this, and that he hoped we would succeed. Furthermore, he let us know that he would like to contact the designer in charge of the exhibitions at the library to create some interesting scenery for the project. Lastly, he expressed that in case we would not succeed to finish the prototype for the beginning of December, and run for the three-week period, he would like to know, so that he could contact the designer and tell her to do some backup exhibition.

\section{Technical point of view}
The basic idea is to have a canvas on which the shade of the person is projected together with the Christmas hat added to the top of the head. For that the following tools are needed:

\begin{itemize}
\item A web camera to gather input about people walking past
\item A computer to run the program code
\item A projector to display the output from the computer
\item A canvas on which the output is projected
\item Controllable light conditions
\item Power, cables, etc.
\end{itemize}

Together with a projector capable of projecting a relatively huge output, 3 * 2.25 meters, the computer running the program code will also be connected to a web cam. A canvas in the above-mentioned size will have to be made and placed on top of the bookshelves.

\subsection{Getting started with a prototype}
The first step into developing the needed software was to write some initial code that would run and produce a useful output. This was done by dividing the group into two smaller groups consisting of three people each, with the goal of coming up with a solution on how to produce a segmentation of a video input. A competition within the group turned up to create the best program code in three days to be presented. And so the "winner's code", or parts from both projects, would be incorporated in a testable file to be elaborated on in the following month.\\
At the end of the week, two programs were ready to be tested. Both programs were able to do same things, only the code differed. Some of the techniques used were: median filtering; morphology (opening/closing); background subtraction; conversion from a colored image to a grayscale image; and, most importantly, a first draft of the BLOB analysis that is needed to analyze the shape of the head where the Christmas hat is to be placed.\\

\subsection{Running the first test}
As the first prototype of the program was completed, it was decided to test code and also the equipment given. Mainly because it was of great importance to check the outcome of the input video, considering the framerate of the video and the threshold values. 

\begin{figure}[htbp]
\centering
\includegraphics[width=1.00\textwidth]{Pictures/Test/TestSetup.jpg}
\caption{Picture from Testing the infrared Camera}
\label{fig:ir_cam_test}
\end{figure} 

Figure \ref{fig:ir_cam_test} illustrates the initial setup for the first prototype testing. \\
At first the ordinary web camera was used, but due to changes in the light conditions from the objects walking past, which caused glittering on the output, it was decided to use the infrared camera. In addition to the infrared camera, four light bulbs were used to illuminate the person properly, so the contrast between background and object became as big as possible. This produced a somewhat good output of the test person, though one problem still remained: once the test person turned 90 degrees, the head became much smaller and didn't look very precise on the output, in addition the Christmas hat would also be misplaced.\\
The solution to this turned out to be illuminating the background instead of the person, which produced a much better result, as shown in figure \ref{fig:max_subtracted}.

\begin{figure}[htbp]
\centering
\includegraphics[width=1.00\textwidth]{Pictures/Test/MaxSubtracted.jpg}
\caption{Complete background subtraction of object outputted as binary picture.}
\label{fig:max_subtracted}
\end{figure}

\section{Feedback from co-supervisor}
The 14th of November a supervisor meeting was arranged with the co-supervisor Andreas M{\o}gelmose. The meeting was initiated to receive some valuable feedback on the prototype test, as well as help with some programming dilemmas regarding speed and efficiency due to the amount of pixels needed to loop through in the program.\\
Various suggestions were given, such as adjusting the BLOB analysis to a more specified point of interest, as well as some tips on how to optimize the background subtraction.

\section{A cartoonish touch}
After the meeting with Andreas, some changes were made to the concept. It was decided that, considering the new information gathered, to take the project in slightly different direction.

\subsection{Using an avatar instead of shape}
Instead of outputting the entire input image (the person itself) and performing several image operations, it was chosen to limit the amount of data used to find the person. Instead of looking at the whole shape, the program should only look for the head. This would save a lot of calculations, so only the head of the user is of interest to analyze. This would hopefully lead to a more stable framerate for the program, which is important to create the proper immersion. It is important that the camera is quick enough to pick up and recognize new people stepping into the scene.

\subsection{Using Unity to handle the graphical output}
The group quickly realized that for it would be difficult to use OpenCV to both extract the data from the camera and display it in some kind of graphical way. Since the program has to loop through a lot of data continuously, there was little resources left for it to actually display the result in an interesting way without lagging behind. First the group thought about using multiple threads running in the program, but even then it would be hard to display more sophisticated graphics. OpenCV is not meant as a tool to display graphics, but more to for the analyzing/extracting part.

Then the group came up with the idea of using the Unity game engine to display the graphics based on the data from OpenCV. This had the added bonus of being able to use elements such as particle effects, physics and sound. Since some group members already had experience with the Unity, it seemed a good choice. The only concern was how to send the data from OpenCV to Unity. OpenCV uses C++, while Unity uses \texttt{C\#}, JavaScript or Boo as scripting languages. Even though there are various APIs to get OpenCV and Unity communicate, it was decided for a much simpler approach where OpenCV would copy some position data into a text file that Unity would read.

In the end, the result should be a smoother experience than if only OpenCV was used.
  
\section{Initial vision}
The vision for the project is to make a simple, yet working, program that is fun and engaging to use at Hj{\o}rring Library. It should be something you can just walk pass once and get something fun out of it, or, if you stay for a little longer, you can interact with it even more. Also, the program should be fairly self-maintained and independent, meaning that little work is needed from the staff of the library.

The group decided that it wanted to make everything from scratch, i.e. not use any existing functionality in OpenCV. This means that even though the program is going to be slower in overall, everything is made by ourselves.

Based on reaching the above-mentioned goals by engaging people at the library, the initial vision of this semester project is to engage visitors, both passively and actively. Ultimately, the final problem statement can be read in \ref{problemStatement}.