\chapter{Conclusion}

In this chapter we will conclude whether or not the project was a success. After having reached a conclusion in the project, we will reflect upon the experiences and knowledge gained during this semester project.

In the beginning of the report a series of problem definitions were listed. Below these are included and we will describe to which degree each of the goals were met.

\textbf{Problem statement:}\\
The group decided that in order to succeed, the program should meet the following sets of requirements:
\begin{itemize}
\item Low maintenance: The program should be easy to use for the library staff.
\item The installation should fit the overall theme of Hj{\o}rring Library during December.
\item It should be entertaining for both active and passive users.
\item Multiple people should be able to use it simultaneously.
\item All of the image processing functionality should be written by the group.
\end{itemize}

The parts from the problem definition will be listed throughout this chapter and commented upon.

\subsubsection{Low maintenance: The program should be easy to use for the library staff}
From the beginning of the project, it was a goal to make the program easy to maintain for the library staff. This was accomplished by limiting the actions required to start the program. The only thing that is needed to turn on the installation is power on three electronic systems. This was explained to the staff in the form of these simple instructions:
\begin{enumerate}
\item Press button 1 - on/off button to turn on the LEDs.
\item Press button 2 - on/off button on the projector.
\item Press button 3 - on/off button on the computer.
\end{enumerate}
Ultimately this should prevent errors and make it possible for the staff and the library to initialize the program without assistance.

\subsubsection{The installation should fit the overall theme of Hj{\o}rring Library during December}
To make the installation fit the theme of Hj{\o}rring Library during December month, Christmas was brought into the project. All the graphical elements support the Christmas theme. As a part of the final testing, visitors of the library were asked whether the installation fit the library. Almost every participant answered that it did indeed fit into the library and the Christmas theme.

%% Mayhaps for perspective? %%%%%  Besides the Christmas theme, Hj{\o}rring Library is known for being modern and appealing to their visitors. Interaction is a keyword at the library, which fits the theme of the third semester perfectly. Considering above-mentioned facts accounted for, it has been concluded that the "Magic Canvas" fits the library and the Christmas theme.

\subsubsection{It should be entertaining for both active and passive users}
Visitors at the library are able to walk past the canvas, spot a character moving, and then decide to engage further or not. The fact that there are multiple characters selected randomly, gives the installation an interesting touch.

\subsubsection{Multiple people should be able to use it simultaneously}
Implementing a way to track multiple persons from frame to frame, makes the program able to represent up to 10 characters at a time.

\subsubsection{All of the image processing functionality should be written by the group}
All image processing functionality was written by the group, the OpenCV library is only used to access pixel values. 

\subsubsection{Concluding experiences}
During our time at the library, both our observations and the information learned from interviews correlated to highlight that the installation was often overlooked. Despite this the project received positive feedback both from the visitors and the library staff, as seen in the attached e-mail.


\begin{fancyquotes}
E-mail December 13, 2012\\
Tak tak, det skal vi nok klare!\\
Og tak for jeres indsats!\\
M{\aa}ske f{\o}ler I at det drukner lidt i alt julehall{\o}jet her hos os, men for os er det en v{\ae}rdifuld erfaring. Som I kan se s{\aa} er de andre juletiltag og udstillinger lidt mere traditionelle og mere til at kigge p{\aa}. Og for os, og brugerne, er det vigtigt at vi er s{\aa} alsidige som muligt, og hele tiden afpr{\o}ver nyt.\\
Med venlig hilsen
Tone Lunden
\end{fancyquotes}



