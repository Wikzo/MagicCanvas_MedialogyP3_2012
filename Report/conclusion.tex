\chapter{Conclusion}

In this chapter we will conclude whether or not we deem the project a success. After having reached a conclusion in the project, we will reflect upon the experiences and knowledge gained during this semester project.

In the beginning of the report a series of problem definitions were listed. Below these are included and we will describe to which degree each of the goals were met.

\textbf{Problem statement:}\\
The group decided that in order to succeed, the program should meet the following sets of requirements:
\begin{itemize}
\item Low maintenance: The program should be easy to use for the library staff.
\item The installation should fit the overall theme of Hj{\o}rring Library during December.
\item The program should be ready at the beginning of December.
\item It should be entertaining for both active and passive users.
\item Multiple people should be able to use it simultaneously.
\item All of the image processing functionality should be written by the group.
\end{itemize}

The parts from the problem definition will be listed throughout this chapter and commented upon.

\subsubsection{Low maintenance: The program should be easy to use for the library staff.}
From the beginning of the project, it was a goal to make the program easy to maintain for the library staff. This was accomplished by limiting the actions required to start the program. The only thing that is needed to turn on the installation is power on three electronic systems. This was explained to the staff in the form of these simple instructions:
\begin{enumerate}
\item Press button 1 - on/off button to turn on the LEDs.
\item Press button 2 - on/off button on the projector.
\item Press button 3 - on/off button on the computer.
\end{enumerate}
Ultimately this should prevent errors and make it possible for the staff and the library to initialize the program without assistance.

\begin{itemize}
\item Fits the theme of Hj{\o}rring Library during December: Christmas. The program should be ready for the beginning of December.
\end{itemize}
To make the installation fit the theme of Hj{\o}rring Library during December month, Christmas was brought into the project. The graphics implemented in Unity; were added to bring and generate a Christmas mood. This was done by implementing some commonly known Christmas icons/symbols - such as snow and Christmas trees. Additionally Christmas characters such as pixies and a snowman were added.\\
As a part of the final testing, visitors of the library were asked whether the installation fit the library. Gratifying for the group almost every participant answered that it did indeed fit in the library and the overall Christmas theme, which is very satisfying for the group. Besides the Christmas theme, Hj{\o}rring Library is known for being modern and appealing to their visitors. Interaction is a keyword at the library, which fits the theme of the third semester perfectly. Considering above-mentioned facts accounted for, it has been concluded that the "Magic Canvas" fits the library and the Christmas theme.


\begin{itemize}
\item It should be engaging for short periods of time, as well as for multiple uses. Multiple people should be able to use it simultaneously.
\end{itemize}
The installation at Hj{\o}rring Library gives the user the liberty to decide themselves, how much time they would like to spend on the installation. There are no requirements needed to use the installation. All one needs is to walk in between the LED-strip, positioned under the red interior of the library and the infrared camera.\\
The user is simply able to walk past the canvas seeing the character move and then decide whether she/he finds it entertaining. The fact that there are multiple characters selected randomly, gives the installation an interesting touch, as there is always a specific character one would rather have. This option is also possible though more than one person is using the installation. However sometimes an error occurs if the users walk too close to each other, as they will be recognized as one person instead of multiple persons.


\begin{itemize}
\item The program would be written in C++ and using the OpenCV library for reading/writing images, but all of the image processing functionality should be written by the group.
\end{itemize}
All image processing functionality was written by the group itself, and the OpenCV library was only used to read and outputting images. However the program was based entirely on C++. All the image processing was written in C++ but the visualisation of the program was done using Unity. It worked by sending coordinates assigned by the C++ programming to Unity through the clipboard on the computer.

\textbf{Is it possible to make an interactive piece of software that fits into the environment of Hj{\o}rring Library?}\\
Throughout this project, it has been proven that it is indeed possible. This project has shown that it is possible to make an interactive piece of software that fits into the environment of Hj{\o}rring library's\\
The submissions gathered through the interviews of test persons proved that Hj{\o}rring library is a great location for such a project. However in terms of not getting overlooked by the visitors, the project needs to be highlighted to create more attention towards it.\\
The project period has been a positive experience for the group, but also for the library. It has been a great experience to work professionally with the library, and it has encouraged the group to work even harder to release a good product. As the pressure lie on your shoulders to meet the deadline and finalize the product one is motivated greatly into working hard.\\
As mentioned above it was a great experience for the library as well. In the mail attached below the library staff reports how this has been a valuable experience for them as well.

\begin{fancyquotes}
E-mail December 13, 2012\\
Tak tak, det skal vi nok klare!\\
Og tak for jeres indsats!\\
M{\aa}ske f{\o}ler I at det drukner lidt i alt jullehall{\o}jet her hos os, men for os er det en v{\ae}rdifuld erfaring. Som I kan se s{\aa} er de andre juletiltag og udstillinger lidt mere traditionelle og mere til at kigge p{\aa}. Og for os, og brugerne, er det vigtigt at vi er {\aa} alsidige som muligt, og hele tiden afpr{\o}ver nyt.\\
Med venlig hilsen
Tone Lunden
\end{fancyquotes}



