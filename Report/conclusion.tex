\chapter{Conclusion}
After the project is finished, it is time to conclude on the experiences and new knowledge that this has provided. In the beginning of the report a problem definition was made and this chapter will conclude on this, showing how the different problems was answered.

Below is the problem definition with answers of how these are taken care of in the project.

\textbf{The group decided that in order to succeed, the program should meet the following sets of requirements:}

\begin{itemize}
\item Low maintenance: the program should be easy set up and use for the library staff.
\end{itemize}
Low maintenance is a key factor, and is important because it shouldn't take much of the library staffs time. This is solved by making it as simple as possible. The only thing that is required from the staff is to press three buttons to turn on the setup. This is explained in a three stepped guide:
\begin{enumerate}
\item Press button 1 - on/off button for electricity for the setup.
\item Press button 2 - on/off button on the projector.
\item Press button 3 - on/off button on the computer.
\end{enumerate}

Once these three steps are completed the computer will start up and automatically open a batch file when Windows boots up. This batch file opens the OpenCV program which will initialize itself. Then after 15 seconds it will automatically open Unity and the magic canvas is ready to use. This being said it takes only two buttons and some seconds to turn it on, which is satisfying for us and the library staff.\\
When turning it off the same three steps are done in reverse order:
 
\begin{enumerate}
\item Press button 1 - on/off button on the computer.
\item Press button 2 - on/off button on the projector (Do this twice to turn it off).
\item Press button 3 - on/off button for electricity for the setup.
\end{enumerate}

\begin{itemize}
\item Fits with the theme of Hj{\o}rring Library during December: Christmas. The program should be ready for the beginning of December.
\end{itemize}
To make the installation fit into the theme of Hj{\o}rring Library during December, the Christmas theme was brought into the project. The graphics should give a Christmas feel, bringing in some of the known Christmas symbols - such as snow and Christmas trees, but also some Christmas characters such as pixies and a snowman.\\
When running the testing visitors of the library was asked whether the installation were fitting inside the library, and almost everyone answered that it did indeed fit into the library and it's theme.

Besides the Christmas theme, Hj{\o}rring Library is known for being very interactive which fits perfect into our theme. With all this being said it can be concluded that the project fits the library and its Christmas theme very well.

\begin{itemize}
\item It should be engaging for short periods of time, as well as for multiple uses. Multiple people should be able to use it simultaneously.
\end{itemize}
The installation gives the opportunity for the user to chose how much time they want to spend on it. There is no ending so it is not required for you to spend a certain amount of time. \\
The user will be able to just walk past the canvas seeing the character move, and then if he/she finds it entertaining then it is possible to play with, walking back and forth controlling the character. The fact that there are multiple characters which comes randomly makes it interesting to play with more than just once.\\
As for 

\begin{itemize}
\item The program would be written in C++ and using the OpenCV library for reading/writing images, but all of the image processing functionality should be written by the group.
\end{itemize}
jkhvbbnm oihj gfgio

\textbf{Is it possible to make an interactive piece of software that fits into the environment of Hj{\o}rring Library?}

\begin{fancyquotes}
E-mail December 13, 2012\\
Tak tak, det skal vi nok klare!\\

Og tak for jeres indsats!\\
M{\aa}ske f{\o}ler I at det drukner lidt i alt jullehall{\o}jet her hos os, men for os er det en v{\ae}rdifuld erfaring. Som I kan se s{\aa} er de andre juletiltag og udstillinger lidt mere traditionelle og mere til at kigge p{\aa}. Og for os, og brugerne, er det vigtigt at vi er {\aa} alsidige som muligt, og hele tiden afpr{\o}ver nyt.\\

Med venlig hilsen
Tone Lunden
\end{fancyquotes}