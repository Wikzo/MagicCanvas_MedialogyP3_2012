\chapter{Perspective}
The main focus of this chapter is to reflect on the P3 project and the decisions made. Furthermore this chapter will account for future implementations that are based on the test results, from the final testing at Hj{\o}rring Library.
\section{Overlooking the installation}
One of the most critical flaws of the installation at Hj{\o}rring Library, was the fact that the majority of the visitors overlooked the installation. After interviewing people walking by the installation it was clear to us that the primary reason for not seeing the installation is due to lack of visual attraction. Few of the people interviewed suggested that we should have a sign saying something like \textbf{"Project by Aalborg University"}. This would give people a chance to notice the installation as they approach the location. In addition some sound effects/background music would also contribute in helping people to notice the installation. At moment the only sounds produced is outputted from the projector and it only appear once Father Christmas appears. An extend to this setup would be to have some active stationary speakers outputting some sound effects from the game in real-time that of course would not bother the staff at the library too much.\\
Considering the visitors who actually noticed the installation. It proved that the active users who engaged had a hard to figure out how to interact with the installation besides from moving from side to side. To accommodate the active users a piece of laminated carton could be designed and placed on the floor beneath the focus point. This should ultimately improve the user experience and give people a reason to play with the installation in regards to using the snowball positioned on the map.\\
\section{Future Implementations}
Considering the feedback gathered from interviewing visitors of Hj{\o}rring Library and the general user testing performed by the group itself, it's clear that some changes are to be made if this installation was ever released. One of the best outcomes of the testing proved to be all the feedback in regards to upgrades that would improve the user experience and the product overall. Especially an improvement of the field of interaction was a request by many.
\subsection{Improved interaction}
One field of interaction wanted by the active users of the installation was in terms of a system that would give them the opportunity to interact with the game by the use of a touch-system. How to implement this could however prove a challenge considering the shades of the person using the installation would be reflected onto the canvas. In theory it would be a great feature to the game that the active users would be able to engage in the installation by touching the surface of the canvas, to activate specific elements of interest to explore.\\
Another edge of the technology could however be an external platform to be managed by the user in extend to the horizontal movement. An example to this could an application build for the Apple iPad. By connecting the laptop running the program to the iPad, one would be able to contribute and engage interactively in the installation on a new level. How this would work in practice is however another issue to address. 
\subsection{Colors/contrast}
One pitfall of the installation was the vague colors outputted by the projector on the IKEA sheet. Due to the light conditions at the library the output from the projector was not very precise and vague, even though 4 lamps had been turned off. Especially bright colors had difficulties to penetrate the light produced by active lamps at the library. A solution to this could be to have darker colors or to even discard the winter landscape, though this could prove other difficulties in continuation of visual attraction and user experience. By discarding graphics, the game could appear less attractive and limit the number of users.
 
\subsection{Frame rate of the output}
As accounted for in \textbf{chapterXX}\fixme{Find reference in document}. The more operations performed in the program code, the lower the frame rate the output will have. Currently the program is running with \textbf{12 fps}\fixme{find reference/precise number}. Producing 12 frames per second is a great progress considering the result from the first prototype testing held at the University, where 3-4 frames were produces each second. However it came to our attention during the testing that one factor, for some people not noticing the installation as they walk by, is the fact that the avatar did not appear before the users were almost past the canvas. Optimizing the frame rate further is however not an easy task, as it would mean that one would have to optimize the existing piece of code or to discard some of the existing material. The primary issue with discarding existing material is the decrease in quality and segmentation of the input. Ultimately the program would not be able to produce a usable and precise output.   
\subsection{New landscapes}
One way to expand the installation and to create more excitement towards using the installation actively, is to create more exciting landscapes, such as "Father Christmas' Workshop". The conceptual idea was that the scenery should change once every day/week with new hourly events added. One idea to Father Christmas' Workshop could be pixie riots where the user would need to assist the foreman.
