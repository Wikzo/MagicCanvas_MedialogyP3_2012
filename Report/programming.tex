\section{Programming in OpenCV and Unity}
\subsection{Using Unity to handle the graphical output}
After doing the initial prototype, the group realized that for it would be difficult to use OpenCV to both extract the data from the camera and display it in some kind of graphical way. Since the program has to loop through a lot of data continuously, there was little resources left for it to actually display the result in an interesting way without lagging behind. First the group thought about using multiple threads running in the program, but even then it would be hard to display more sophisticated graphics. OpenCV is not meant as a tool to display graphics, but more to for the analyzing/extracting part.

Then the group came up with the idea of using the Unity game engine to display the graphics based on the data from OpenCV. This had the added bonus of being able to use elements such as particle effects, physics and sound. Since some group members already had experience with the Unity, it seemed a good choice. The only concern was how to send the data from OpenCV to Unity. OpenCV uses C++, while Unity uses \texttt{C\#}, JavaScript or Boo as scripting languages. Even though there are various APIs to get OpenCV and Unity communicate, it was decided for a much simpler approach where OpenCV would copy some position data into a text file that Unity would read.

In the end, the result should be a smoother experience than if only OpenCV was used.

Here is how we did stuff

What we did and why?

\subsection{Our own picture class}

\subsection{Config Background}

\subsection{The four in one function}
discard information that is not in the ROI(Make the pixels black that aren't used)
refresh(Open in openCV, write to write to pointer system)
background subtraction(compare the current picture to the background)
threshold pixel values and set to true(255) or false(0)  

\subsection{Morphology}
Closing
	erode
	dilate
	
\subsection{Heysa}
Different possibilities (pros and cons)
Show code and describe features
Test individual features - e.g. why did we use threshold value X instead of Y?

OpenCV
Using Picture struct
ROI
Using ClipBoard Manager
Unity - uses time of day to change stuff
\subsection{Unity}